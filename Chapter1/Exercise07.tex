Let $X$ be a vector space of all complex functions on the unit interval $[0,1]$, topologized by the family of seminorms
$$
p_x(f)=|f(x)|, \quad x\in [0,1].
$$
This topology is called the **topology of pointwise convergence**. Justify this terminology
\begin{itemize}
    \item A sequence converges pointwise if for every $x \in [0,1]$, the sequence of complex numbers $(f_n(x))$ converges in $\mathbb{C}$. That is to say for $x$ and $\varepsilon>0$, there exists $N=N(x,\varepsilon)$ such that $n\geq N$ implies $|f_n(x)-f(x)|<\varepsilon$. Here the seminorm $p_x(f_n(x)-f(x))$ converges pointwise to $0$ as $n$ increases.
\end{itemize}